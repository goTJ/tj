% vim:fileencodings=big5,latin1:ft=tex:foldmethod=marker
% {{{ setting
\documentclass[12pt,a4paper]{article}
\renewcommand{\baselinestretch}{1.2}
\pagestyle{empty}
\usepackage[margin=2.5cm]{geometry}
\usepackage{CJK}
\usepackage{color}
\usepackage{amsmath,amsthm,amssymb}
\usepackage{graphicx}
\theoremstyle{remark}
\newtheorem{thm}{Theorem}[section]
\newtheorem{defi}{Define}[section] %}}}
% {{{ header
\begin{CJK}{Bg5}{ming}
\title{VOIP HW2 Document}
\author{b92902106 �����}%\thanks{�p�������ij, �бH�� flyhermit@csie.org, ����}}
\date{\today}
\end{CJK} % }}}

\begin{document}
% {{{ title
\begin{CJK}{Bg5}{ming}
\maketitle
%\tableofcontents
\end{CJK} % }}}
% {{{ content
\begin{CJK}{Bg5}{ming}
% {{{ \begin{abstract}
\begin{abstract}
This is a simple location server and redirect server of SIP/UDP. Which is implemented
by C++ programming language and unix socket programming. It can handle normal situation and
in case of request sip address not found when redirecting(send 404 Not Found).
\end{abstract} % }}}
% {{{ \section{Installation}
\section{Installation}
\subsection{Requirement}
The server only runs on unix-like machines, and it's well tested on linux.
To install it, we need:
\begin{itemize}
\item Hardware:
	\begin{itemize}
		\item An unix-like machine on Internet with real IP.
	\end{itemize}
\item Software:
	\begin{itemize}
		\item tar \& gzip (For extracting files)
		\item Berkeley socket library. (It's usually included on every unix-like machine.)
		\item g++
		\item make
	\end{itemize}
\end{itemize}

\subsection{Step by Step}
\subsubsection*{To install it}
\begin{enumerate}
\item \$ tar xzvf reg\_redir\_server.tgz   \# Extract files
\item \$ cd reg\_redir\_server.tgz   \# Change directory to reg\_redir\_server
\item \$ make   \# Compile
\end{enumerate}
\subsubsection*{To excute it}
\begin{enumerate}
\item \$ ./server
\end{enumerate}
It will set up server and listen UDP packets to port 5060.
\subsubsection*{To clean it}
\begin{enumerate}
\item \$ make clean
\end{enumerate}
It will clean up all object files.

\subsection{Testing}
I only have tested server with CCL SIP UA(Ver. 1.02 Build 4) and it worked normally.
But I can't promiss it'll perform well with all SIP UA.
% }}}
% {{{ \section{Design}
\section{Design}
\subsection{Class Intro}
There are four classes:
\begin{itemize}
	\item SockIO: A class manages socket IO and transmits/receives messages on UDP.
	\item RegData: A structure stores whole register infomation.
	\item RegDB: A class communicates with register database. It associates with RegData.
	\item SipReq: A class reads and judges all SIP requests, and then sends responses.
		It depends on SockIO and is generalized by RegData.
\end{itemize}
% {{{ Classes
\subsubsection{SockIO Class}
The public methods are listed below.
\begin{verbatim}
class SockIO{
public:
        SockIO();
        ~SockIO();
        bool setIn(const int port);
        bool setOut(const string &ip, const int port);
        bool setOutByLast(); // Set out ip/port
                             // by ip/port of last reading data
        bool write(const string &); // Write data to out-ip/port.
        string read(); // Read data from in-port.
                       // If no data comes, it'll block.
};
\end{verbatim}
\subsubsection{RegData Class}
The public methods and protected data are listed below.
\begin{verbatim}
class RegData{
public:
        friend class RegDB;
        RegData(){};
        RegData(const RegData &r);
        void eraseData(); // Erase whole data in the object.
protected:
        string _to; // The information of request address
        string _contact; // The information of register address
                         // mapping to request address.
        string _expires; // The information of Expires field.
        time_t _expireTime; // The expire time of this data.
};
\end{verbatim}
\subsubsection{RegDB Class}
The public methods are listed below.
\begin{verbatim}
class RegDB{
public:
        bool insert(RegData &); // Insert given RegData into DB.
        bool query(RegData &); // Query RegData from DB to given
                               // RegData reference..
};
\end{verbatim}
\subsubsection{SipReq Class}
The public methods are listed below.
\begin{verbatim}
class SipReq: public RegData{
public:
        enum Method{UNKNOWN, ACK, REGISTER, INVITE}; // Define types
                                                     // of method.
        SipReq(const int port); // Set up server with given port.
        void getHeader(); // Get request header.
                          // If no header comes, it'll block.
        void eraseHeader(); // Erase all field of header.
        Method getMethod(); // Return the type of method of last coming
                            // request.
        void send(const int code, const string &description);
                 // Send response with given code number and description.
};
\end{verbatim}
% }}}
\subsection{Main Function}
There are two objects in Main function: sipReq and regDB. sipReq is an object
of SipReq class, and it sets up location/redirect server.
Then regDB is an object of RegDB class, and it provides interface to store
register data. The mechanism of Main function is very simple. A while loop runs
infinitely and the sipReq waits request. When request comming, classify it by
methods and judges it only by header information. So it's a stateless server.
The main flow is listed below.
\begin{verbatim}
while(1){ // Run infinitely.
    sipReq.getHeader();
    switch(sipReq.getMethod()){
        case SipReq::REGISTER:
            regDB.insert(sipReq) ? sipReq.send(200, description[200])
                                 : sipReq.send(404, description[404]);
            break;
        case SipReq::INVITE:
            regDB.query(sipReq) ? sipReq.send(302, description[302])
                                : sipReq.send(404, description[404]);
            break;
        case SipReq::ACK:
            break;
        default:
    }
    sipReq.eraseHeader();
}
\end{verbatim}
% }}}
\end{CJK} % }}}
\end{document}
