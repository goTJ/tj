% vim:fileencodings=big5:ft=tex:foldmethod=marker
% {{{ setting
\documentclass[12pt,a4paper]{article}
\renewcommand{\baselinestretch}{1.5}
\pagestyle{empty}
\usepackage[margin=2.5cm]{geometry}
\usepackage{CJK}
\usepackage{color}
\usepackage{amsmath,amsthm,amssymb}
\usepackage{graphicx}
\theoremstyle{remark}
\newtheorem{thm}{Theorem}[section]
\newtheorem{defi}{Define}[section] %}}}
% {{{ header
\begin{CJK}{Bg5}{bsmi}
\title{Summary}
\author{R96922011 �����(Tien-Jung Chuang)}%\thanks{�p�������ij, �бH�� flyhermit@gmail.com, ����}}
\end{CJK}
\date{\today} % }}}

\begin{document}
% {{{ title
\begin{CJK}{Bg5}{bsmi}
\maketitle
%\tableofcontents
\end{CJK} % }}}
\begin{CJK}{Bg5}{bsmi}

\section{Method} % {{{
The purpose of this homework is a general shot detector,
so I prefer to detect shots only by the variance between two adjoining frames.
A simple, efficient, and general approach is to compare frames with color histogram,
which has low sensitivity to camera and object movement rather pixel-pair comparison.
Therefore, I decide to use this method to compare frames.

But I think there are still some ambiguous situations for shot detection
to use color histogram comparison.
Roughly I classify the variance between two adjoining frames into six classes:

\begin{center}
\begin{tabular}{|c|c|c|c|}
\hline
& Small change & Medium change & Large change \\
\hline
Scene change & & Gradual transition & Sharp transition \\
\hline
No scene change & Object motion & Camera movement & \\
\hline
\end{tabular}
\end{center}

We can notice that it's hard to separate gradual transition and object motion easily.
One way to improve it is to compare features with structure feature,
such as edge, texture, and feature point.
But I think most feature extractors don't maintain consistence in sequence images
or hard to implement. So instead I use region histogram, split image into multiple regions,
comparing variance in each region, and discarding the regions with the largest variance.
It'll reduce the effects of camera movement, in consequence we can increase the precision.

Our purpose is a general shot detector. But usually we need to set threshold.
It's hard to find a threshold suiting for all video.
One approach is to normalize the variance by some magic formula.
I prefer to eliminate the threshold. I use K-means to cluster variance into 2-class,
"scene change" and "no scene change."
I think it's a very great approach but still has some limitations.
As I refered above, the change has three levels, but K-means can only have fixed class number.
If the video has no transition or has three levels change,
2-class K-means may make some mistakes.
I try to modify it by iterative K-means, but it didn't work well.

To sum up, I use K-means to detect scene change, using color histogram comparison.
I've tested variance in different color spaces(RGB, HSV, YIQ),
different distance functions(L1, X2-test), w/ or w/o using regions histogram.
The definitions of L1 and X2-test are below: ($H$ means hisogram)
\[
	X^2 = \sum_{bins}\frac{(H_{k+1}-H_{k})^2}{(H_(k+1)+H_{k})^2}
\]
\[
	L1 = \sum_{bins}|H_{k+1}-H_{k}|
\]
% }}}

\section{Result} % {{{
After testing, YIQ color space with X2-test region histogram works best.
It can detect 01, 03, and 04 great without changing threshold (Answer question D)
and do well on 02. (Answer C: there are some several smooth gradual transitions on 02.)
So in my program I use this method.
% }}}

\section{Reference} % {{{
\begin{enumerate}
\item J.S. Boreczky, L.A. Rowe, "Comparison of video shot boundary detection techniques," Proc of SPIE- Storage and Retrieval for Still Image and Video Databases IV, Vol. 2670, San Diego, 1996
\item I. Koprinska, S. Carrato, "Temporal video segmentation: a survey," Signal Processing: Image Communication, vol. 16, pp. 477--500, 2001.
\item B. Gunsel , M. Ferman and A. M. Tekalp ��Temporal video segmentation using unsupervised clustering and semantic object tracking,�� J. Electron. Imag., vol. 7, pp. 592, Jul. 1998.
\end{enumerate}
% }}}
\end{CJK}
\end{document}
