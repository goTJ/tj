% vim:fileencodings=big5:ft=tex:foldmethod=marker
% {{{ setting
\documentclass[12pt,a4paper]{article}
\renewcommand{\baselinestretch}{1.2}
\pagestyle{empty}
\usepackage[margin=2.5cm]{geometry}
\usepackage{CJK}
\usepackage{color}
\usepackage{amsmath,amsthm,amssymb}
\usepackage{graphicx}
\theoremstyle{remark}
\newtheorem{thm}{Theorem}[section]
\newtheorem{defi}{Define}[section] %}}}
% {{{ header
\begin{CJK}{Bg5}{bsmi}
\title{MMAI HW2 - CBIR}
\author{R96922011 �����(Tien-Jung Chuang)}%\thanks{�p�������ij, �бH�� flyhermit@gmail.com, ����}}
\end{CJK}
\date{\today} % }}}

\begin{document}
% {{{ title
\begin{CJK}{Bg5}{bsmi}
\maketitle
%\tableofcontents
\end{CJK} % }}}
\begin{CJK}{Bg5}{bsmi}
\begin{abstract} % {{{
This program provides three query methods: color, texture, and hybrid methods.
It uses global color histogram in HSV color space as color feature,
gabor texture as texture feature, and L1 distance to calculate the similarity.
In addition, this program uses leave-one-out method to train the fusing parameter
to achieve a better result.
\end{abstract} % }}}
\section*{Introduction} % {{{
\subsection*{Programs} % {{{
\begin{itemize}
\item gen\_fv.m: it generates features of images in \textit{list.txt}
		and stores them into \textit{fvs.mat}.
\item query.m: it queries images in \textit{list\_query.txt} from \textit{fvs.mat}
		, and stores the results into \textit{result.html} in HTML format.
\end{itemize}
% }}}
\subsection*{Functions} % {{{
\begin{itemize}
\item fv\_color.m: it reads a imagename and returns the color features of the image.
\item fv\_texture.m: it reads a imagename and returns the texture features of the image.
\item gaborconvolve.m: it reads a gray-level image matrix
	and returns images convolved by gabor filter.
\item fv\_sim.m: it reads two feature vectors and returns the distance between them.
\item im\_query.m: it reads a imagename and returns the query results.
\end{itemize}
% }}}
% }}}
\section*{Methods} % {{{
\subsection*{Color Feature} % {{{
I transform the color space of image to HSV and quantify H:S:V channels to 18:3:3 parts.
Then I calculate the whole image into one 162-bins histogram
and normalize the histogram due to different sizes of images.\\

Besides, I didn't split image into different regions because
I think in CBIR there are different view points of scenes,
so intuitively it won't improve the accuracy much
but increase the dimensions of features(curse of high dimensions!!).
Therefore, I leave the comparison of local features to the texture features.
% }}}
\subsection*{Texture Feature} % {{{
I transform the image to gray-level and convole it by Gabor filter(4 scales and 6 orientations).
Then calculate means and variances of the magnitudes of all convolutions.
\\

The reason of I choosing Gabor feature is that it's a killer-leveled texture feature.
It bases on Fourier feature which can generally extract texture features for most images
and provides local spatial frquency analysis to improve to disadvantage of Fourier feature.
% }}}
\subsection*{Similarity} % {{{
I just simply use L1 distance to calculate the similarity for color and texture features.
L2 is also a simple method, but I think it's easy to cause the curse of high dimensions.
In addition, when combining two features, we have to set the weighting of two features.
Therefore, I define a fusing parameter \textit{color\_p}($0 \leq color\_p \leq 1$),
and the similarity of hybrid feature is:
\[
	color\_p*dis_{color}(fv1_{color}, fv2_{color}) + (1-color\_p)*dis_{texture}(fv1_{texture}, fv2_{texture})
\]
where $dis_{color}$ and $dis_{texture}$ is the distance function of color and texture,
and $fv1_{color}$ and $fv2_{color}$ represent two color feature vectors,
similar as $fv1_{texture}$ and $fv2_{texture}$.
\\

To find a suitable $color\_p$, I use leave-one-out method to tune the parameter.
I query all image in the dataset except for itself,
calculate the distance by different $color\_p$, sum the total hits among all top 9 images,
and choose the $color\_p$ with the most total hits.
% }}}
% }}}
\section*{Result} % {{{
Please see \textit{result.html}.
% }}}
\section*{Discussions} % {{{
\subsection*{Color Features} % {{{
The color features only compares with the global color distribution.
Therefore images with similar color will get high scores.
Of course it's undoubted that the result contains many similar color images.
% }}}
\subsection*{Texture Features} % {{{
I think there are two main factors may affect accuracy,
one is whether the query image has a obvious pattern and
the other is if any similar pattern image is in the dataset.
For example, there is no specific pattern of baseball fields,
so the result contains many different type of images.
In the other hand, follows and foods both have similar patterns,
same as the memorial tower and the statue of liberty.
Therefore, the results contain those similar images.
% }}}
\subsection*{Hybrid Features} % {{{
The results of hybrid features don't look good.
There are several reasons.
First, only color and texture features can't describe images well,
some classes have semantic meanings or
just refer an object in a image(the background will be noise).
Second, the global fusing parameter may not be a good choice,
every classes have different properties,
some has much color meaning and some has much texture meaning.
The recent research proposes the idea of mixed kernel
which assigns differnt fusing parameters for each class.
It seems able to increase the accuracy.
Third, the way of training fusing parameters is also a key factor.
I calculated the sum the number of top 9 hits
because I only focused on high ranking results.
Another way is to calculate the area of PR curves,
it'll maximize the average accuracy.
% }}}
\end{CJK}
\end{document}
