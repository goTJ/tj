% vim:fileencodings=big5,latin1:ft=tex:foldmethod=marker
% {{{ \documentclass[12pt,a4paper]{report}
\documentclass[12pt,a4paper]{report}
\oddsidemargin=0cm
\topmargin=0cm
\hoffset=0cm
\voffset=0mm
\marginparwidth=0cm
\textwidth=15cm
\pagestyle{empty}
\usepackage{CJK}
\usepackage{color}
\usepackage{amsmath,amsthm,amssymb}
\usepackage{clrscode}
\theoremstyle{remark}
\newtheorem{thm}{Theorem}[section]
\newtheorem{defi}{Define}[section]

\begin{CJK}{Bg5}{ming}
\title{}
\author{�����(Tien-Jung Chuang)\thanks{�p�������ij, �бH�� flyhermit@csie.org, ����}}
\date{\today}
\end{CJK}
% }}} \documentclass[12pt,a4paper]{report}

\begin{document}
\begin{CJK}{Bg5}{ming}
%\maketitle
%\tableofcontents
\end{CJK}

\begin{CJK}{Bg5}{ming}

\begin{flushright}
�ջy�@�~�@\\
��u�G b92902106 �����
\end{flushright}

% {{{ \subsubsection{Pseudo Code}
\subsubsection{Pseudo Code}
Here is a pseudo code of bubble sort:\\
\begin{codebox}
\Procname{$\proc{Bubble-Sort}(data)$}
\zi \For $i \gets 1$ \To $\id{length}[data]$
\zi 	\Do \For $j \gets 2$ \To $\id{length}][data]$
\zi 		\Do \If $data[j-1] > data[j]$
\zi			\Then exchange $data[j-1] \leftrightarrow data[j]$
		\End
	\End
    \End
\zi display data
\end{codebox}
% }}} \subsubsection{Pseudo Code}

% {{{ \subsubsection{Documentation}
\subsubsection{Documentation}
This program implements bubble sort(showed above) by nasm.
I use cmp+jmp to implement loop and if, and I add some comments in source code,
I think it's easy to understand this short code.
% }}} \subsubsection{Documentation}

% {{{ \subsubsection{How to assemble}
\subsubsection{How to assemble}
I assembled it on Unix-like machine(linux6), using {\sf nasm} and {\sf gcc} because I called
{\sl printf} and {\sl puts} for outputing. The following line shows how to assemble it:
\begin{verbatim}
	$ nasm -f elf hw1.asm           # generate object file
	$ gcc hw1.o -o hw1              # link files and generate excutable file
\end{verbatim}
% }}} \subsubsection{How to assemble}

% {{{ \subsubsection{How to excute}
\subsubsection{How to excute}
Also I excuted it on Unix-like machine(linux6), The following lines shows how to excute it:
\begin{verbatim}
	$ ./hw1
\end{verbatim}
% }}} \subsubsection{How to excute}

% {{{ \subsubsection{Input format}
\subsubsection{Input format}
	No input needs.
% }}} \subsubsection{Input format}

% {{{ \subsubsection{Output format}
\subsubsection{Output format}
	Sorting result.
% }}} \subsubsection{Output format}

% {{{ \subsubsection{Bonus}
\subsubsection{Bonus}
Please see file {\sf hw1\_bonus.asm}. Here is brief steps to implement qsort:\\
\begin{codebox}
\zi \While stack is not empty
\zi	\Do Pop stack
\zi 	Partition data
\zi 	Push subarray if it's not empty.
    \End
\zi display data
\end{codebox}
~\\
The Usage is similar to hw1.asm.
% }}} \subsubsection{Bonus}

\end{CJK}
\end{document}
