% vim:fileencodings=big5,latin1:ft=tex:foldmethod=marker
% {{{ setting
\documentclass[12pt,a4paper]{article}
\renewcommand{\baselinestretch}{1.2}
\pagestyle{empty}
\usepackage[margin=2.5cm]{geometry}
\usepackage{CJK}
\usepackage{color}
\usepackage{amsmath,amsthm,amssymb}
\usepackage{graphicx}
\theoremstyle{remark}
\newtheorem{thm}{Theorem}[section]
\newtheorem{defi}{Define}[section] %}}}
% {{{ header
\begin{CJK}{Bg5}{ming}
\title{Compiler HW1}
\author{B92902106 �����(Tien-Jung Chuang)}%\thanks{�p�������ij, �бH�� flyhermit@csie.org, ����}}
\date{\today}
\end{CJK} % }}}

\begin{document}
% {{{ title
\begin{CJK}{Bg5}{ming}
\maketitle
%\tableofcontents
\end{CJK} % }}}
% {{{ content
\begin{CJK}{Bg5}{ming}
\subsubsection*{3.10(c)}
Yes. For each $r\{m, n\}$, we can replace it with concatenating $m$ times of $r$ and $n-m$
times of $r?$. For example, $r\{2,5\} \equiv rrr?r?r?$.
\subsubsection*{3.10(d)}
No. RE can't reference both previous and next characters(it's context-free).
In addition, now we consider about LEX RE.
For each $r\$$, we can replace it with $r/[\backslash n]$.
And for each [\textasciicircum $A$], where $A$ is a set of characters,
we can replace it with [$\bar{A}$], where $\bar{A}$ is the complement of $A$.
But for each \textasciicircum$r$, there is no RE that could reference previous input,
so it's impossible to replace it. (In addition, for LEX, it's possible to use BEGIN to do it.)\\
Therefore, we can't replace it with neither RE nor LEX RE.
\end{CJK} % }}}
\end{document}
