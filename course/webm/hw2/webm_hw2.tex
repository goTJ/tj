% vim:fileencodings=big5:ft=tex:foldmethod=marker
%Report

%The report (from 1 to 3 pages) should cover the following sections.

%* Implementation for each task
	%o The tools/code you used in developing each component
	%o The detailed parameter settings
%* Credits (for each member)

%A few lines of comments to this assignment are also highly welcome. 
% {{{ setting
\documentclass[12pt,a4paper]{article}
\renewcommand{\baselinestretch}{1.2}
\pagestyle{empty}
\usepackage[margin=2.5cm]{geometry}
\usepackage{CJK}
\usepackage{color}
\usepackage{amsmath,amsthm,amssymb}
\usepackage{graphicx}
\theoremstyle{remark}
\newtheorem{thm}{Theorem}[section]
\newtheorem{defi}{Define}[section] %}}}
% {{{ header
\begin{CJK}{Bg5}{bsmi}
\title{Web Mining HW2 Report}
\author{R96922011 �����\thanks{�����: r96011@csie.ntu.edu.tw}�BR96922012 ��޳�P\thanks{��޳�P: r96012@csie.ntu.edu.tw}�BR96922125 �H���\thanks{�H���: r96125@csie.ntu.edu.tw}}
\end{CJK}
\date{\today} % }}}

\begin{document}
% {{{ title
\begin{CJK}{Bg5}{bsmi}
\maketitle
%\tableofcontents
\end{CJK} % }}}
\begin{CJK}{Bg5}{bsmi}
To make description clear, we can divide this project into three parts:
\begin{enumerate}
	\item Collecting documents and extracting terms from those documents
	\item Generating features of each document
	\item Training model
\end{enumerate}
And we'll discuss eash part in each section.

\section{Collections} % {{{
It's almost dirty work here, we collected web pages from urllist given in assignment web page
	and some crawled pages of HW1(dropped too few words pages and inaccessible pages).
First, we converted web page into plain text by ignoring all html tags, comments, etc..
Then we converted Chinese words to Big5 encoding and stored them as numbers.
For English, we just extracted letter strings and separated combined word(with a '-') into two words.
All is writen by Perl, because Perl is good at it.
Finally we transfered web page into terms sequences.
% }}}
\section{Generating Features} % {{{
We regarded a bigram of Chinese and a unigram of English as one dimension of features
	, and we dropped the bigrams or the unigram appearing less than 3 times.
The dimension of features is about 200,000.
To build features of document, we considered the existence of bigram and unigram.
If it exists in a document, we set this dimension 1. Otherwise, we set 0.
We implemented it in C++ for high efficiency.
% }}}
\section{Training Model} % {{{
We used linear svm to train model because the features suit for linear character and we could speed up our training.
Therefore, we could afford 200,000 dimensions features.
% }}}
\section{Credits} % {{{
Collections: ��޳�P\\
Generating Features: �H���\\
Training Model: �����\\
Interface: �����
% }}}
\end{CJK}
\end{document}
